\appendix
\chapter*{Appendix A: Glossary}
\addcontentsline{toc}{chapter}{Appendix B: Glossary}
\label{appendix:glossary}

\textbf{Headless CMS:} A content management system that provides content through APIs without a traditional frontend. The content is stored and managed in the backend but can be displayed on any device or platform through API calls.

\textbf{JSON:API:} A specification for building APIs in JSON format that defines how resources should be formatted, how relationships should be handled, and how data should be queried. It provides a standardized way to structure API responses.

\textbf{Monorepo:} A software development strategy where multiple related projects are stored in a single repository. This approach facilitates code sharing, dependency management, and coordinated development across projects.

\textbf{Slug:} A user-friendly URL identifier derived from a page title or content name. 

\textbf{UUID (Universally Unique Identifier):} A 128-bit number used to uniquely identify information in computer systems. In Drupal, UUIDs are used to identify content entities across different environments.

\textbf{Widget:} A reusable component that encapsulates specific functionality and can be embedded in pages or other widgets. In this project, widgets bridge Drupal content management with Next.js component rendering.

\textbf{Content Type:} A predefined template in Drupal that defines the structure and fields for a specific type of content. Examples include "Job Offer," "Candidature," or "Quiz." Each content type can have custom fields and behaviors.

\textbf{Entity:} The fundamental unit of content in Drupal. Entities can be nodes (content), users, taxonomy terms, or custom entities. Each entity has fields that store data and can have relationships with other entities.


\textbf{Hook:} A PHP function that allows modules to interact with Drupal core and other modules. Hooks follow a naming convention (hook\_event\_name) and are called at specific points in Drupal's execution to allow custom functionality.

\textbf{Paragraph:} A Drupal entity type that allows content creators to build flexible, structured content by combining different paragraph types. Each paragraph type can have different fields and layouts.

\textbf{Taxonomy:} A system for classifying and organizing content using terms and vocabularies. For example, a "Specialization" vocabulary might contain terms like "Frontend Development," "Backend Development," and "DevOps."

\textbf{Vocabulary:} A collection of taxonomy terms used for categorizing content. Multiple vocabularies can exist for different classification systems (e.g., Tags, Categories, Specializations).

\textbf{Webform:} A Drupal module that provides form building capabilities, allowing administrators to create complex forms with various field types, validation rules, and submission handling logic.


\textbf{AccessHandler Controller:} A Drupal class responsible for determining whether a user has permission to perform specific operations (view, edit, delete) on entities. It implements access control logic based on user roles and permissions.


\textbf{Webpack:} A module bundler for JavaScript applications that processes and optimizes assets (JavaScript, CSS, images) for production deployment. It can also run custom plugins during the build process.
