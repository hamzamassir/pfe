\chapter{Discussion and Future Improvements}
\label{chap:technical_challenges}

\section{Introduction}
\noindent
This chapter evaluates the Intern Management Platform developed for VOID Digital Agency. We examine the current results, identify limitations, and propose future improvements. The platform has transformed how VOID manages interns, but there are opportunities for enhancement.

\section{Current Achievements}

\noindent
The Intern Management Platform has successfully improved VOID's intern management process. The system handles intern onboarding, project assignments, and performance tracking in a centralized manner. The transition from manual processes to an automated system has reduced administrative workload and improved efficiency.

The technical architecture using Drupal 10 backend with Next.js frontend has proven effective. The system handles multiple concurrent users and provides reliable performance. The component-driven development approach has made maintenance easier and ensured consistent user interface across the platform.

User experience has improved significantly. Interns can easily access their project information, track their progress, and communicate with supervisors. The recruitment team reports better organization and reduced time spent on routine administrative tasks.

\section{Current Limitations}

\noindent
The platform has several areas that need improvement. The administrative dashboard is complex and requires technical knowledge to use effectively. Non-technical staff find it difficult to navigate and perform basic tasks. The interface lacks modern design elements that would improve usability.

The intern workspace has limitations in collaboration features. Interns cannot easily work together on shared tasks or projects. There is no real-time communication system for team collaboration during sprints. The platform lacks tools for group project management and peer-to-peer interaction.

Technical infrastructure needs enhancement. The current system does not provide virtual development environments for interns. This limits their ability to work on complex projects that require specific technical setups. The platform also lacks real-time assistance features that could help interns during their work.

\section{Future Improvements}

\subsection{VPS Integration for Development Environments}

\noindent
We plan to integrate Virtual Private Server (VPS) technology to provide dedicated development environments for interns. This will allow each intern to have their own isolated workspace with the necessary tools and configurations for their projects. The VPS integration will support multiple programming languages and frameworks, enabling interns to work on diverse technical projects without environment conflicts.

The VPS system will include automated setup and teardown processes. When an intern joins a project, their development environment will be automatically configured with the required dependencies and access permissions. This will reduce setup time and ensure consistency across all intern workstations.

\subsection{Enhanced Administrative Dashboard}

\noindent
We will develop a new, user-friendly administrative dashboard specifically designed for non-technical users. The new interface will feature simplified navigation with intuitive menu structures. Administrators will have role-based access control, ensuring they only see relevant information and functions.

The dashboard will include visual workflow management tools. Administrators will be able to drag and drop intern assignments and track project progress through visual representations. Customizable widgets will allow different administrative roles to personalize their dashboard according to their specific needs.

\subsection{Real-Time Assistance in Sprints}

\noindent
We will implement real-time assistance features to support interns during sprint development cycles. The system will include instant messaging capabilities for quick communication between interns and supervisors. Real-time notifications will alert interns about task updates, deadline reminders, and important announcements.

The platform will feature integrated help systems and documentation access. Interns will have immediate access to project guidelines, coding standards, and troubleshooting resources. This will reduce the time spent searching for information and improve productivity during development sprints.

\subsection{Collaborative Tasking for Group Projects}

\noindent
We will enhance the platform with collaborative tasking features to improve group collaboration between interns. The system will support shared task boards where interns can view, assign, and track group project tasks. Real-time updates will ensure all team members stay informed about project progress.

The collaborative features will include shared code repositories, document collaboration tools, and peer review systems. Interns will be able to work together on the same projects simultaneously, with version control and conflict resolution mechanisms. This will foster teamwork and improve the quality of group deliverables.

\section{Impact and Benefits}

\noindent
These improvements will significantly enhance the intern experience at VOID Digital Agency. The VPS integration will provide interns with professional development environments, preparing them for real-world technical work. The enhanced dashboard will reduce administrative overhead and improve operational efficiency.

Real-time assistance features will increase intern productivity and reduce the learning curve for new team members. Collaborative tasking will improve teamwork skills and project outcomes. These enhancements will position VOID as an attractive destination for talented interns seeking comprehensive learning experiences.

\section{Summary}

\noindent
The Intern Management Platform has successfully improved VOID's intern management processes. While the current system provides solid foundation, the proposed improvements will significantly enhance the platform's capabilities. The focus on VPS integration, improved dashboard design, real-time assistance, and collaborative features will create a more comprehensive and user-friendly intern management solution.

These enhancements will not only improve current operations but also position VOID for future growth. The platform's modular architecture allows for incremental implementation of these features, ensuring minimal disruption to ongoing operations while delivering continuous value to both interns and administrative staff.