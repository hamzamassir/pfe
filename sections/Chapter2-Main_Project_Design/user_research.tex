\section{User Research and Methodology}
\noindent
To ensure the Intern Management Platform would genuinely address the needs of its users, a thorough user research phase was conducted at the outset of the project. This phase combined direct engagement with stakeholders, analysis of existing workflows, and benchmarking against industry best practices. The insights gained were instrumental in shaping both the functional and technical design of the platform.

\subsection{Research Methods}
\begin{itemize}
    \item \textbf{Stakeholder Interviews:} In-depth interviews were held with HR staff, technical leads, former interns (us, and our colleagues). These conversations provided a nuanced understanding of the pain points, expectations, and day-to-day realities of each group involved in the internship process.
    \item \textbf{User Journey Mapping:} The team mapped out the complete candidate journey—from discovering the program on the main website, through application, technical testing, challenge submission, video interview, onboarding, and training. This exercise highlighted critical touchpoints and potential friction points.
\end{itemize}

\subsection{User Journey Analysis}
The research revealed several critical stages in the candidate and intern journey that required special attention:
\begin{itemize}
    \item \textbf{Application Phase:} Candidates needed a simple, intuitive process for discovering opportunities, submitting applications, and uploading documents. Real-time status tracking was essential to reduce anxiety and uncertainty.
    \item \textbf{Selection Process:} Automated technical tests, challenge submissions, and video interviews were identified as effective ways to assess candidate skills and motivation. Clear communication of next steps and timely feedback were crucial for maintaining engagement.
    \item \textbf{Internship Period:} Once accepted, interns valued having a dedicated training space with structured learning paths, progress tracking, and easy access to resources. Supervisors needed tools to monitor progress, provide feedback, and manage evaluations efficiently.
    \item \textbf{Evaluation:} Both interns and supervisors required a streamlined process for performance assessment, documentation, and certification, with the ability to track outcomes and support future opportunities.
\end{itemize}


