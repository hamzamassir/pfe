\chapter{Reflection}
\label{ch:reflection}

\noindent
Working on this internship at VOID Digital Agency taught me way more than I expected when I first walked through their doors. What started as tackling their messy recruitment process—they were drowning in over 600 applications every year—turned into building something that actually makes a difference in how they work.

The whole experience showed me what it really means to solve real problems with code. Before this, most of my projects were academic exercises. Here, I had to think about actual people using what I built: stressed candidates checking their application status at midnight, HR folks trying to manage hundreds of applications without losing their minds, and technical reviewers who needed quick ways to evaluate coding challenges.

Building with Drupal 10 and Next.js was definitely a learning curve. I'd worked with React before, but never in this kind of setup where the backend and frontend are completely separate. Getting comfortable with JSON:API took time, and I probably spent way too many late nights figuring out why my API calls weren't working. But once it clicked, I could see why this architecture makes so much sense for complex applications.

The Docker stuff was completely new territory for me. I remember being intimidated by all those configuration files and wondering if I'd ever understand what was actually happening when I ran \texttt{docker-compose up}. Now I'm the guy explaining containers to my classmates. Setting up the CI/CD pipeline with Bitbucket was another challenge that pushed me out of my comfort zone, but seeing code automatically deploy after passing tests felt pretty amazing. Oh, and I also discovered that my old laptop wasn't quite up for the job—turns out I really need to invest in a MacBook Pro if I want to keep doing this kind of work without waiting five minutes for Docker to start up.

What I'm most proud of isn't just the technical side though. The system actually works for real people doing real work. Candidates can now apply without sending emails back and forth, complete their technical tests online, and know exactly where they stand in the process. The HR team went from juggling spreadsheets and email threads to having everything organized in one place. That's the kind of impact I want my code to have.

This project changed how I think about software development. It's not just about making things work; it's about making them work well for the people who actually use them. The feedback sessions with candidates and staff taught me more about user experience than any textbook could. Sometimes the most elegant technical solution isn't the right one if it confuses the end user.

Looking back, I wish I'd spent more time early on really understanding the existing workflow before jumping into coding. I also learned that documentation isn't just busy work—future me (and my teammates) really appreciate it when trying to understand why certain decisions were made.

This internship proved that I can handle real-world development challenges, work with production systems, and build software that people actually depend on. More importantly, it showed me what kind of developer I want to become: someone who writes code that solves real problems and makes people's work lives a little bit easier.
